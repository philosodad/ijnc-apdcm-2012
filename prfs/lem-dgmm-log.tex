\begin{lem}
\label{lem:dgmm-log}
Algorithm~\ref{alg:dgmm} (DMMW) terminates in $O(log \Delta)$ rounds with high probability for random connected graphs.
\end{lem}


\begin{proof}[Proof of Lemma~\ref{lem:dgmm-log}]
\begin{ldef}
A node is {\em committed} if it has joined the cover or if all of its neighbors have joined the cover.
\end{ldef}
\begin{ldef}
A node is {\em active} if it is not committed.
\end{ldef}
\begin{ldef}
$\Delta$ is the maximum degree of the Graph.
\end{ldef}

\begin{ldef}
The active degree of a node $u$ is the number of unweighted edges of $u$ indicated by $\alpha(u)$. $\alpha(u)$ is either 0 for committed nodes or equal to the number of active neighbors of $u$ for active nodes.
\end{ldef}
\begin{ldef}
$\delta$ is the largest active degree in the graph.
\end{ldef}

Lemma~\ref{lem:dgmm-log} can be restated in terms of the following propositions:
\begin{lprp}
\label{prop:dgmm-log-each}
Each active node in the graph joins the cover with a constant probability in each round.
\end{lprp}
\begin{lprp}
\label{prop:dgmm-log-alpha}
With high probability, $\delta$ decreases by a constant fraction in each round.
\end{lprp}
\begin{proof}[Proof of Proposition~\ref{prop:dgmm-log-each}]

We begin with a node, $w$, which is an active node in the graph. We want to show that the probability that $w$ will join the cover is constant. Let us assume that $w$ chooses to be a receiver, an event which happens with $P(0.5)$

We know that $w$ has some number of active neighbors, indicated by $\alpha(w)$. Each neighbor of $w$ also has some number of active neighbors. Each active degree is an integer between 1 and $\delta$. We can consider each node to have approximately the same active degree, which we indicate by $\alpha$. 

Each active neighbor of $w$ also chooses to be a sender or a receiver with equal probability. Following the assumption of equal distribution, there are $\sfrac{\alpha}{2}$ senders in the neighborhood of $w$, and each sender has $\alpha$ neighbors.

Each neighbor of $w$ that chooses to be a sender will, according to Line~\ref{alglin:dgmm-issue-invite} of Algorithm~\ref{alg:dgmm}, choose one neighbor which is active and send an invitation to that node. Because there are approximately $\sfrac{\alpha}{2}$ inviting neighbors of $w$, we can consider the probability that $w$ will recieve an invitation to be approximately equivalent to the probability that any event with a probability of $\sfrac{1}{n}$ will occur in $\sfrac{n}{2}$ independent trials for $n > 0$. 

Therefore, the probability that $w$ will recieve an invitation from an active neighbor is: \begin{equation}
1 - \left(\frac{n-1}{n}\right)^{\frac{n}{2}} > \lim_{0 \to \infty} 1 - \left(\frac{k-1}{k}\right)^{\frac{k}{2}} = 1 - \frac{1}{\sqrt{{\mathrm{e}}}} \approx 0.393
\end{equation}


If $w$ recieves at least one invitation, it will respond to exactly one invitation (\algref{alg:dgmm}{alglin:dgmm-choose-invite}), so the probability that $w$ will respond to an invitation from some active neighbor $v$ is exactly the same as the probability that $w$ will recieve an invitation from some active neighbor $v$. Therefore, if $w$ is a receiver, $w$ joins the cover with a probability of $1 - \sfrac{1}{\sqrt{{\mathrm{e}}}}$, which is constant. The probability of $w$ being a receiver at all is $\sfrac{1}{2}$, and the probability of $w$ joining the cover is also $\sfrac{1}{2}$, as we assume that weights are distributed arbitrarily through the graph. Therefore, the probability that $w$ will join the cover in any given round is constant.
\end{proof}

\begin{proof}[Proof of Proposition~\ref{prop:dgmm-log-alpha}]

We consider a node $u$, where $\alpha(u) = \delta$. $u$ may choose to be a sender or a receiver in this round, and $u$ may or may not join the matching. We do make the assumption that $u$ does not join the cover in this round.

$u$ has $\delta$ active neighbors. According to Proposition~\ref{prop:dgmm-log-each}, each of these neighbors joins the cover with constant probability. So in round one, we expect some constant percentage $p$ of the neighbors of $u$ to join the cover, and some constant percentage $q = 1-p$ of the neighbors of $u$ to remain active. Therefore, the value of $\delta(n+1)$ (where $n+1$ represents a round number) is $\delta(n) \times q$. Therefore $\alpha$ decreases at a constant rate, which is what we wanted to show.

\end{proof}

\begin{cor}WHP, the number of communication rounds required to resolve Algorithm~\ref{alg:dgmm} for a random graph is $O(\log\Delta)$.\end{cor}

\end{proof} 
