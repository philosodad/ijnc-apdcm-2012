\begin{lem}
\label{lem:edge-color-terminate}
Algorithm~\ref{alg:edge-color} is likely to terminate in $O(\Delta)$ rounds.
\end{lem}

\begin{proof}[Proof of Lemma~\ref{lem:edge-color-terminate}]

Algorithm~\ref{alg:edge-color} terminates when all of the nodes have colored all of their edges. In order to color an edge, a node must form a pair in an a given round with each of its neighbors. The number of neighbors of each node is $\delta \le \Delta$.

We note that there is no limit on the number of nodes that can participate in any given round. If a particular graph has a complete matching, than every compute node may participate in a round. Further, the fact that a node is participating guarantees that at least one of its neighbors is participating, but does not prevent any of its other neighbors from participating as well.

Therefore, if the probability that a given node $w$ will participate in the computation for a given round can be shown to be constant, than the number of compute rounds is bounded by that constant times $\Delta$.

We therefore will procede to find the probability that a random node $w$ will color one of its edges in some round $r$.

Following the same reasoning as the proof of Lemma~\ref{lem:dgmm-log}, we know that in any given round, a which is a receiver will receive an invitation with a constant probability, as shown in Equation~\ref{eqn:bootstrap}. Nodes choose to be a receiver with constant probability, and any node which is a receiver and receives at least one invitation will respond to one invitation.

Therefore the probability that a node will be in the matching in a given round is constant.

A node in the matching will color at least one edge, and no node has more than $\Delta$ edges, so Algorithm~\ref{alg:edge-color} will terminate in $O(\Delta)$ rounds.

\end{proof}
