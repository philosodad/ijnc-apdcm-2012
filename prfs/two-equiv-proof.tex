

\begin{thm}
  Algorithm~\ref{alg:dhgmm} produces a 2-approximate edge cover.
  \label{thm:dhgmm}
\end{thm}
\begin{proof}{Proof of Theorem~\ref{thm:dhgmm}}
  \begin{smy}
    The outline of the proof is as follows. Algorithm~\ref{alg:dhgmm} produces a cover of the hypergraph $G$, which we will define as \cAd. We would like to show that \cAd\ is a 2-approximation of \cCd\, the optimal cover of $G$. In order to show that this is the case, we divide the appoximate cover into two non-overlapping sets. The first set, \bId, conists of the vertices shared by \cAd\ and \cCd\. Because \bId\ is a subset of \cCd\, we know that the weight of \bId\ is less than or equal to the weight of \cCd. 

The second set, \bOd\, consists of vertices that are in the approximate cover, but not in the optimal cover. We would like to show that the weight of \bOd\ is less than or equal to the weight of \cCd. If it is, than the weight of the approximate cover is no greater than twice the weight of the optimal cover, and we have proved that \cAd\ is a 2-approximation of \cCd\, which is what we want to show. This can be proven by using the {\em incident weight} of the vertices, as defined in section~\ref{sub:alg-dgmm}. 

\cAd\ can be divided into two sets, \bId\ and \bOd, that is, those vertices that are in \cAd\ and also in the optimal edge cover, and those vertices that are in \cAd\ but are not in the optimal edge cover. In the formal proof below, the optimal cover is referred to by \cCd. \footnote{The mnemonics for this proof are as follows: we are using calligraphic letters to refer to the {\em Approximate} cover and the optimal {\em Cover}. The approximate cover is divided into the two sets denoted by blackboard style, those vertices that are {\em In} the optimal cover and those that are {\em Out} of the optimal cover. Weight of any set or vertex is denoted with an $\omega$ because of the similarity to w on the one hand, but also because the symbol $\varpi$ looks like a w plus an edge, which conceptually relates to the concept of {\em incident weight}. Incidence is illustrated by placing a line over the incident vertices.} What we would like to show is that both the set of vertices that are in both \cAd\ and \cCd\ and the set of vertices that are exclusive to \cAd\ have a total weight less than or equal to that of the optimal cover. If this is true, than the sum of the weights of the two sets is less than or equal to twice the weight of the optimal cover, and Algorithm~\ref{alg:dhgmm} produces a 2-approximate cover.
  \end{smy}
    
  \begin{defn}
    \begin{description}
    \item[$G$] A hypergraph $G(V,E)$ with vertices $v \in V$ and edges $e\{u,v,...,z\} \in E$
    \item[\cCd] The optimal edge cover of $G$
    \item[\cAd] The cover generated by Algorithm~\ref{alg:dhgmm}
    \item[\bId] $v \in \cA \land v\in \cC$
    \item[\bOd] $v \in \cA \land v\notin \cC$
    \item[$\omega_u$] The weight of u
    \item[$\varpi_v$] The incident weight of v
    \item[$ \overline{uv} $] u is incident to v in $G$
    \end{description}
  \end{defn}
  \begin{prop}
    $\omega_{\bI} \le \omega_{\cC} \land \omega_{\bO} \le \omega_{\cC} \implies \omega_{\cA} \le 2(\omega_{\cC})$, which is what must be shown as explained in the summary.
    \label{prp:weightcomp}
  \end{prop}

  \begin{lem}
    $\omega_{\bI} \le \omega_{\cC}$
    \label{lem:IleC}
  \end{lem}
  \begin{proof}{Proof of Lemma~\ref{lem:IleC}}
    This is obvious, as no vertex can be counted twice, and every vertex that is in \bId\ is also in \cCd. So the sum of the weights of the vertices that are in both the generated cover and the optimal cover is less than the sum of the weights of the optimal cover.
  \end{proof}
  \begin{lem}
    $\omega_{\bO} \le \omega_{\cC}$
    \label{lem:OleC}
  \end{lem}
  \begin{note}
    Lemma~\ref{lem:OleC} states that the sum of the weights of the vertices that are in the Aproximate cover but not in the optimal cover is less than or equal to the sum of the weights of the optimal cover.
  \end{note}
  \begin{lem}
    $\forall u \in C$, $\sum_{\omega_v} (v \in \bO\text{, }\overline{vu}) \le \omega_u$
    \label{lem:incidentweight}
  \end{lem}
  \begin{note}
    Lemma~\ref{lem:incidentweight} specifies Lemma~\ref{lem:OleC} to a single vertex in the optimal cover. 
  \end{note} 
  \begin{proof}{Proof of Lemma~\ref{lem:OleC},~\ref{lem:incidentweight}}
    \begin{equation}
      v \in \bO \land v \notin \cC \implies \overline{uv} \text{, } u \in \cC
    \end{equation}
    Assume Lemma~\ref{lem:incidentweight} is false. In this case, $\varpi_u \le 0$. However, in order for this to happen, a vertex $v\sigma u$ would have been chosen by Algorithm~\ref{alg:dhgmm} when it was both being compared to $u$ and $\varpi_v > \varpi_u$. This contradicts the basic function of the algorithm, Lemma~\ref{lem:incidentweight} is true.
    Lemma~\ref{lem:incidentweight} implies $\omega_{\bO} \le \omega_{\cC}$, so Lemma~\ref{lem:OleC} is true as well.
  \end{proof}
        
  Therefore, the conditions of proposition~\ref{prp:weightcomp} are satisfied, and Thereom~\ref{thm:dhgmm} is true.
\end{proof}
