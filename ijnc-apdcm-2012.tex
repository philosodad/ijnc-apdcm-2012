\documentclass[twoside]{article}
\usepackage{verbatim}
\usepackage{multirow} \usepackage{enumerate}
\usepackage{amsmath,enumerate} \usepackage{amsthm}
\usepackage{algcompatible}
\usepackage{algpseudocode}
\usepackage{algorithm}
%\usepackage{algorithmic}
%\usepackage{pstricks}
\usepackage{amssymb, latexsym}
\usepackage{xfrac}
\usepackage{mathtools}
\usepackage{graphicx}
\usepackage[captionskip=5pt, nearskip=5pt, font=small]{subfig}
\DeclareGraphicsRule{*}{mps}{*}{}
\usepackage{listings}

%specific to this document only
%\usepackage{pgfplots}
\usepackage{pgfplotstable}
\pgfplotstableread{plts/experiment_erdren_edgecolor.tab}\erdrenone
\pgfplotstableread{plts/experiment_erdren_edgecolor_400.tab}\erdrentwo
\pgfplotstableread{plts/experiment_erdren_directededgecolor.tab}\erddirone
\pgfplotstableread{plts/experiment_erdren_directededgecolor_400.tab}\erddirtwo
\pgfplotstableread{plts/experiment_sclfre_edgecolor.tab}\sclfreeone
\pgfplotstableread{plts/experiment_sclfre_edgecolor_400.tab}\sclfreetwo
\pgfplotstableread{plts/experiment_smlwld_edgecolor.tab}\smlwldone
\pgfplotstableread{plts/experiment_smlwld_edgecolor_64.tab}\smlwldtwo
\pgfplotstableread{plts/experiment_smlwld_edgecolor_256.tab}\smlwldthree
%\pgfplotstableread{plts/experiment8b2_av.tab}\averagetwo
%\pgfplotstableread{plts/experiment8b3_av.tab}\averagethree
%\pgfplotstableread{plts/experiment8b4_av.tab}\averagefour
%\pgfplotstableread{plts/experiment9a_av.tab}\stepping
%\pgfplotstableread{plts/experiment9a1_av.tab}\steppingone
%\pgfplotstableread{plts/experiment9a2_av.tab}\steppingtwo
%\pgfplotstableread{plts/experiment9a3_av.tab}\steppingthree
%\pgfplotstableread{plts/experiment9a4_av.tab}\steppingfour
%\pgfplotstableread{plts/experiment9b1_av.tab}\runningone
%\pgfplotstableread{plts/experiment9b2_av.tab}\runningtwo
%\pgfplotstableread{plts/experiment9b3_av.tab}\runningthree
%\pgfplotstableread{plts/experiment9b4_av.tab}\runningfour
%\pgfplotstableread{plts/experiment9b_av.tab}\running
%\pgfplotstableread{plts/experiment9c_av.tab}\costcomp
%\pgfplotstableread{plts/experiment9c1_av.tab}\costcompone
%\pgfplotstableread{plts/experiment9c2_av.tab}\costcomptwo
%\pgfplotstableread{plts/experiment9c3_av.tab}\costcompthree
%\pgfplotstableread{plts/experiment8b1_rn.tab}\runsone
%\pgfplotstableread{plts/experiment8b2_rn.tab}\runstwo
%\pgfplotstableread{plts/experiment8b3_rn.tab}\runsthree
%\pgfplotstableread{plts/experiment8b4_rn.tab}\runsfour
%\pgfplotstableset{
%  create on use/density/.style={
%    create col/expr={\thisrow{nodes}+\thisrow{links}}}
%    }
\pgfplotstableset{
  create on use/delta/.style={
    create col/expr={\thisrow{links}/\thisrow{nodes}}
    }}
%\pgfplotstableset{
%  create on use/nodebylinks/.style={
%    create col/expr={(\thisrow{nodes}*\thisrow{links})}}
%    }
%\pgfplotscreateplotcyclelist{three}{% 
%  every mark/.append style={fill=teal}\\% 
%  every mark/.append style={fill=green}\\% 
%  every mark/.append style={fill=orange}\\% 
%}
%\pgfplotscreateplotcyclelist{four}{%
%  every mark/.append style={fill=teal}\\%
%  every mark/.append style={fill=green}\\%
%  every mark/.append style={fill=orange}\\%
%  every mark/.append style={fill=pink}\\%
%}
%\pgfplotscreateplotcyclelist{three-1-0}{%
%  every mark/.append style={fill=teal}\\% 
%  every mark/.append style={fill=green}\\% 
%  every mark/.append style={fill=orange}\\%
%	every mark/.append style={fill=none}\\% 
%	every mark/.append style={fill=none}\\% 
%	every mark/.append style={fill=none}\\% 
%}
%\pgfplotscreateplotcyclelist{three-0-1}{%
%	every mark/.append style={fill=none}\\% 
%	every mark/.append style={fill=none}\\% 
%	every mark/.append style={fill=none}\\% 
%  every mark/.append style={fill=teal}\\% 
%  every mark/.append style={fill=green}\\% 
%  every mark/.append style={fill=orange}\\%
%}
%\pgfplotscreateplotcyclelist{four-1-0}{%
%  every mark/.append style={fill=teal}\\%
%  every mark/.append style={fill=green}\\%
%  every mark/.append style={fill=orange}\\%
%  every mark/.append style={fill=pink}\\%
%	every mark/.append style={fill=none}\\%
%	every mark/.append style={fill=none}\\%
%	every mark/.append style={fill=none}\\%
%	every mark/.append style={fill=none}\\%
%}
%\pgfplotscreateplotcyclelist{four-0-1}{%
%	every mark/.append style={fill=none}\\%
%	every mark/.append style={fill=none}\\%
%	every mark/.append style={fill=none}\\%
%	every mark/.append style={fill=none}\\%
%  every mark/.append style={fill=teal}\\%
%  every mark/.append style={fill=green}\\%
%  every mark/.append style={fill=orange}\\%
%  every mark/.append style={fill=pink}\\%
%}

%%%%%%%%%%%%%

\usepackage{pgf}
\usepackage{tikz}
\usetikzlibrary{decorations.pathmorphing} % LATEX and plain TEX when using Tik Z
\usetikzlibrary{positioning}
\usetikzlibrary{er}
\usetikzlibrary{automata}
\usetikzlibrary{shapes.geometric}
\usetikzlibrary{shapes.misc}
\tikzstyle{vx}=[draw,circle,fill=white,minimum size=2pt, inner sep=1pt, node distance=15mm]
\tikzstyle{ex}=[draw,rectangle,fill=white,minimum size=2pt, inner sep=3pt, node distance=15mm]
\tikzstyle{nfo}=[anchor=north west, rectangle, fill=white,text width=4cm, inner sep=3pt, node distance=15mm]
\tikzstyle{bup}=[semithick, decoration={bent, aspect=.3, amplitude=4}, decorate, ->, >=stealth]
\tikzstyle{bdn}=[semithick, decoration={bent, aspect=.3, amplitude=-4}, decorate, ->, >=stealth]
\tikzstyle{BUP}=[thick, decoration={bent, aspect=.3, amplitude=8}, decorate, ->, >=stealth]
\tikzstyle{BDN}=[thick, decoration={bent, aspect=.3, amplitude=-8}, decorate, ->, >=stealth]
\tikzstyle{MUP}=[thick, decoration={bent, aspect=.3, amplitude=16}, decorate, ->, >=stealth]
\tikzstyle{MDN}=[thick, decoration={bent, aspect=.3, amplitude=-16}, decorate, ->, >=stealth]
\tikzstyle{pln}=[draw, dotted]
\tikzstyle{dir}=[draw, dotted, decorate, ->, >=stealth, bend left=10]
\tikzstyle{str}=[semithick, decorate, ->, >=stealth]
\tikzstyle{cr}=[draw, circle, fill=black!25,minimum size=150pt]
\tikzstyle{rst}=[state, shape=rounded rectangle, rounded rectangle arc length=90, text width=2cm, inner sep=4pt]
\tikzstyle{rsf}=[state, fill=green!35, shape=rounded rectangle, rounded rectangle arc length=90, text width=1.75cm, inner sep=4pt]
%styles for plots?
\tikzstyle{bls}=[blue, solid, mark=square*]
\tikzstyle{grt}=[red, solid, mark=*]
\tikzstyle{inv}=[draw=none]

% \paperheight=11in \paperwidth=8.5in \textheight=9.0in
% \textwidth=6.5in \voffset=-.875in \hoffset=-.875in
\newenvironment{code} {\begin {quote}\begin{footnotesize}}
    {\end{footnotesize}\end{quote}}

% \oddsidemargin 0.0 in \evensidemargin 0.0 in
\newenvironment{enumeratealpha}
{\begin{enumerate}[(a{\textup{)}}]}{\end{enumerate}}

\theoremstyle{plain}
\newtheorem{lem-rule}{Rule}
\newtheorem{thm}{Theorem}
\newtheorem{con}{Conjecture}
\newtheorem{lem}{Lemma}[thm]
\newtheorem{prp}{Proposition}
\newtheorem{prop}{Proposition}[con]
\newtheorem{lprp}{Proposition}[lem]
\newtheorem{cor}{Corollary}[lem]
\theoremstyle{definition}
\newtheorem{defn}{Definition}[thm]
\newtheorem{defi}{Definition}
\newtheorem{dfn}{Definitions}[thm]
\newtheorem{ldef}{Definition}[lem]
\newtheorem{assm}{Assumption}
\theoremstyle{remark}
\newtheorem*{smy}{Summary}
\newtheorem{note}{Note}[thm]
\newtheorem{case}{Case}
%algorithms commands
\algblockdefx[Case]{Case}{EndCase} %
[1] [{\em var}] {{\bfseries case} {\em #1\ } } %
{{\bfseries end case}}%
\algcblockdefx[Case]{Case}{When}{EndCase}
[1] [{\em true}] {{\bfseries when} {\em #1\ }}
{{\bfseries end case}} %

\algblockdefx[TimesDo] {DoTimes}{EndTimes}
[1] [0] {#1 times {\bfseries do}}
{{\bfseries end do}}

%subalgorithms environment
\makeatletter
\newcounter{parentalgorithm}
\newenvironment{subalgorithms}{%
%  \refstepcounter{algorithm}%
  \floatname{algorithm}{Procedure}
  \protected@edef\theparentalgorithm{\thealgorithm}%
  \setcounter{parentalgorithm}{\value{algorithm}}%
  \setcounter{algorithm}{0}%
  \def\thealgorithm{\theparentalgorithm-\alph{algorithm}}%
  \ignorespaces
}{%
  \setcounter{algorithm}{\value{parentalgorithm}}%
  \ignorespacesafterend
}
\makeatother

%code environments
\usepackage{float}
 
\floatstyle{ruled}
\newfloat{codeblock}{thp}{lop}
\floatname{codeblock}{Example}

\lstnewenvironment{rubyblock} 
{\lstset{language=Ruby, breaklines=true, basicstyle=\small, xleftmargin=10pt, numbers=left, numberstyle=\tiny, stepnumber=1, numbersep=5pt, escapeinside={|;}{;|}}}
{}
% text macros
\def\cI{{\mathcal I}} \def\cR{{\mathcal R}} \def\cE{{\mathcal E}}
\def\cC{{\mathcal C}} \def\cF{{\mathcal F}} \def\cU{{\mathcal U}}
\def\cH{{\mathcal H}} \def\cD{{\mathcal D}} \def\cB{{\mathcal B}}
\def\cQ{{\mathcal Q}} \def\cV{{\mathcal V}} \def\cS{{\mathcal S}}
\def\cG{{\mathcal G}} \def\cA{{\mathcal A}} \def\cO{{\mathcal O}}
\def\cW{{\mathcal W}} \def\cL{{\mathcal L}} 

\def\bI{{\mathbb I}} \def\bO{{\mathbb O}}
\def\bC{{\mathbb C}} \def\bM{{\mathbb M}}
\def\bId{{$\mathbb I$}} \def\bOd{{$\mathbb O$}}
\def\bCd{{$\mathbb C$}} \def\bMd{{$\mathbb M$}}

\def\cId{{$\mathcal I$}} \def\cRd{{$\mathcal R$}} \def\cEd{{$\mathcal E$}} 
\def\cCd{{$\mathcal C$}} \def\cFd{{$\mathcal F$}} \def\cUd{{$\mathcal U$}} 
\def\cHd{{$\mathcal H$}} \def\cDd{{$\mathcal D$}} \def\cBd{{$\mathcal B$}} 
\def\cQd{{$\mathcal Q$}} \def\cVd{{$\mathcal V$}} \def\cSd{{$\mathcal S$}} 
\def\cGd{{$\mathcal G$}} \def\cAd{{$\mathcal A$}} \def\cOd{{$\mathcal O$}}
\def\cWd{{$\mathcal W$}} \def\cLd{{$\mathcal L$}}

\def\suchthat{{\: |\:}}




\usepackage{IJNC}


\setcounter{page}{501}
\newcommand{\jvolume}{X}
\newcommand{\jnumber}{Y}
\newcommand{\jmonth}{January}
\newcommand{\jyear}{20XX}

% title
\newcommand{\jtitle}{A General Purpose Automata for Distributed Graph Algorithms}
% running title for header
\newcommand{\jrunningtitle}{A General Purpose Automata}



\pagestyle{plain}


\begin{document}
\thispagestyle{empty}
\copyrightheader

\begin{center}
% print title
\jtitle

\vspace{20pt}

J. Paul Daigle 

\vspace{2pt}

Department of Computer Science, Georgia State University\\
Atlanta, GA, 30303, USA

\vspace{10pt}
%and
%
\vspace{10pt}
Sushil K. Prasad

\vspace{2pt}
Department of Computer Science, Georgia State University\\
Atlanta, GA, 30303, USA

\vspace{10pt}
%and
%
%\vspace{10pt}
%THIRD AUTHOR
%
%\vspace{2pt}
%Group, Laboratory, Address\\
%City, State ZIP, Country

\vspace{20pt}
\publisher{(received date)}{(revised date)}{(accepted date)}{Editor's name}
%\publisherA{(received date)}{(revised date)}{(revised date)}{(accepted date)}{Editor's name}

\end{center}


\begin{abstract}
We present an automata for a compute node in a message passing model which allows a distributed computer to approximate solutions for three NP-Complete graph problems. Our algorithm for the Vertex Cover Problem uses $O(\log{\Delta)}$ communication rounds to find a 2-approximate solution. Our edge coloring algorithm is also 2-approximate and resolves in $O(\Delta)$ rounds, and we find a correct strong edge coloring of a symmetric digraph in $O(\Delta)$ rounds. All three algorithms require only one hop information to find correct solutions. 
\keywords{Distributed Algorithms, Vertex Cover, Edge Coloring, Strong Edge Coloring}
\end{abstract}


\section{Introduction}

In this paper we'll show the automata itself, then our results for vertex cover and edge color.

\subsection{Definitions}

\begin{defi}[Minimum Vertex Cover]
\label{sub:mvc}

Given an undirected Graph $G(V,E)$, a {\em Vertex Cover} of $G$ is a set of vertices $V'$ such that for each edge $e_{u,v} \in E$, $u \in V'$ or $v \in V'$. The Minimum Vertex Cover Problem is to find the smallest possible vertex cover of $G$.

\end{defi}

\begin{defi}[Minimum Weighted Vertex Cover]
\label{sub:mwvc}
Given an undirected Graph $G(V,E)$, where each $v \in V$ has a positive weight $w(v)$, minimize $\sum_{v \in V'} w(v)$.
\end{defi}

\begin{defi}[Edge Coloring]
\label{def:ec}
An edge coloring of a graph is an assignment of colors to the edges of a graph in such a way that no two adjacent edges are assigned the same color. 

Formally, given a graph, $G(V,E)$, and set of colors $C$, an edge coloring of $G$ is a mapping $f(e) = E \mapsto C \suchthat \forall\, e(u,v), e'(v,w) \in E, c \in C, f(e) = c \implies f(e') \ne c$.

\end{defi}


\begin{defi}[Strong Directed Edge Coloring]

A strong edge coloring is an assignment of colors to the edges of the graph such that no two edges that can be connected by a common edge are assigned the same color. 

In the directed case, a strong edge coloring is a mapping:
  \begin{align*} 
    f(e) = C \mapsto E \suchthat \forall\, e(u,v), e'(v,u), e''(w,v), e'''(w,x) \in E, c \in C,\\ 
    f(e) = c \implies f(e') \ne c, f(e'') \ne c, f(e''') \ne c 
  \end{align*} 
\end{defi}


\subsection{Prior Work}

\subsubsection{Vertex Cover}

Sequential Linear time algorithms for covering problems are surveyed in detail in \cite{254190}. The seminal paper on Linear Programming techniques for constant ratio approximation of MWVC was published by Bar-Yehuda and Even in 1981 \cite{Bar-Yehuda:1981lr}. Gonzalez created a 2-approximate LP-Free linear time algorithm based on Maximal Matching in 1995 which is the basis of our distributed algorithm \cite{Gonzalez1995129}. 

We are aware of two prior distributed algorithms for minimum weighted vertex cover. Grandoni et. al present a 2-approximate algorithm based on maximal matching in 2008\cite{1435381}. This algorithm uses $O(\log n + \hat{W})$ communication rounds, where $\hat{W}$ is the average vertex weight, and $n$ is the number of vertexes. {\AA}strand and Suomela presented a $O(\Delta + \hat{W})$ deterministic algorithm in 2010 which is also 2-approximate. Koufogiannakis and Young present a simpler $O(\log n)$ algorithm in 2009 \cite{1582746}. 

Parnas and Ron argue in 2007 that the growth of a randomized algorithm for solving MWVC in a message passing model is linear with the average degree of the graph\cite{Parnas:2007:AMV:1280283.1280327}. Following that work, Onak, Ron, Rosen and Rubinfeld present a deterministic algorithm that 2-approximate in $O(\delta)$ time, where $\delta$ is the average degree of the graph\cite{Onak:2012:NSA:2095116.2095204}. Our own experimental results and algorithm proof appear to conflict with this work, possibly because of differnces between the models of computing. 

\subsubsection{Edge Coloring}

\section{The Vertex Cover Problem}

In this section we'll present the findings for Vertex Cover.

\subsection{Algorithm}

In this section we'll show the algorithm for vertex cover.

\subsection{Analysis}

\label{ssb:algorithms-dgmm-performance}

Despite its overall simplicity, Algorithm~\ref{alg:dgmm} has a running time of $O(\log \Delta)$, which improves on the previous best result of $O(\log n)$ in \cite{1582746} without sacrificing the performance bound of two-optimality. A formal proof of our performance claims follows.

\begin{thm}
  Algorithm~\ref{alg:dgmm} (DGMM) will generate a 2-optimal cover in $O(log \Delta)$ communication rounds with high probability on random input.
\label{thm:dgmm-term}
\end{thm}

\begin{proof}[Proof of Theorem~\ref{thm:dgmm-term}]
\label{prf:correct}

\begin{lem}
\label{lem:dgmm-edge}
  DGMM weights each edge once in a manner equivalent to GMM.
\end{lem}


\begin{lem}
\label{lem:dgmm-log}
Algorithm~\ref{alg:dgmm} (DMMW) terminates in $O(log \Delta)$ rounds with high probability for random connected graphs.
\end{lem}


\begin{proof}[Proof of Lemma~\ref{lem:dgmm-log}]
\begin{ldef}
A node is {\em committed} if it has joined the cover or if all of its neighbors have joined the cover.
\end{ldef}
\begin{ldef}
A node is {\em active} if it is not committed.
\end{ldef}
\begin{ldef}
$\Delta$ is the maximum degree of the Graph.
\end{ldef}

\begin{ldef}
The active degree of a node $u$ is the number of unweighted edges of $u$ indicated by $\alpha(u)$. $\alpha(u)$ is either 0 for committed nodes or equal to the number of active neighbors of $u$ for active nodes.
\end{ldef}
\begin{ldef}
$\delta$ is the largest active degree in the graph.
\end{ldef}

Lemma~\ref{lem:dgmm-log} can be restated in terms of the following propositions:
\begin{lprp}
\label{prop:dgmm-log-each}
Each active node in the graph joins the cover with a constant probability in each round.
\end{lprp}
\begin{lprp}
\label{prop:dgmm-log-alpha}
With high probability, $\delta$ decreases by a constant fraction in each round.
\end{lprp}
\begin{proof}[Proof of Proposition~\ref{prop:dgmm-log-each}]

We begin with a node, $w$, which is an active node in the graph. We want to show that the probability that $w$ will join the cover is constant. Let us assume that $w$ chooses to be a receiver, an event which happens with $P(0.5)$

We know that $w$ has some number of active neighbors, indicated by $\alpha(w)$. Each neighbor of $w$ also has some number of active neighbors. Each active degree is an integer between 1 and $\delta$. We can consider each node to have approximately the same active degree, which we indicate by $\alpha$. 

Each active neighbor of $w$ also chooses to be a sender or a receiver with equal probability. Following the assumption of equal distribution, there are $\sfrac{\alpha}{2}$ senders in the neighborhood of $w$, and each sender has $\alpha$ neighbors.

Each neighbor of $w$ that chooses to be a sender will, according to Line~\ref{alglin:dgmm-issue-invite} of Algorithm~\ref{alg:dgmm}, choose one neighbor which is active and send an invitation to that node. Because there are approximately $\sfrac{\alpha}{2}$ inviting neighbors of $w$, we can consider the probability that $w$ will recieve an invitation to be approximately equivalent to the probability that any event with a probability of $\sfrac{1}{n}$ will occur in $\sfrac{n}{2}$ independent trials for $n > 0$. 

Therefore, the probability that $w$ will recieve an invitation from an active neighbor is: \begin{equation}
1 - \left(\frac{n-1}{n}\right)^{\frac{n}{2}} > \lim_{0 \to \infty} 1 - \left(\frac{k-1}{k}\right)^{\frac{k}{2}} = 1 - \frac{1}{\sqrt{{\mathrm{e}}}} \approx 0.393
\end{equation}


If $w$ recieves at least one invitation, it will respond to exactly one invitation (\algref{alg:dgmm}{alglin:dgmm-choose-invite}), so the probability that $w$ will respond to an invitation from some active neighbor $v$ is exactly the same as the probability that $w$ will recieve an invitation from some active neighbor $v$. Therefore, if $w$ is a receiver, $w$ joins the cover with a probability of $1 - \sfrac{1}{\sqrt{{\mathrm{e}}}}$, which is constant. The probability of $w$ being a receiver at all is $\sfrac{1}{2}$, and the probability of $w$ joining the cover is also $\sfrac{1}{2}$, as we assume that weights are distributed arbitrarily through the graph. Therefore, the probability that $w$ will join the cover in any given round is constant.
\end{proof}

\begin{proof}[Proof of Proposition~\ref{prop:dgmm-log-alpha}]

We consider a node $u$, where $\alpha(u) = \delta$. $u$ may choose to be a sender or a receiver in this round, and $u$ may or may not join the matching. We do make the assumption that $u$ does not join the cover in this round.

$u$ has $\delta$ active neighbors. According to Proposition~\ref{prop:dgmm-log-each}, each of these neighbors joins the cover with constant probability. So in round one, we expect some constant percentage $p$ of the neighbors of $u$ to join the cover, and some constant percentage $q = 1-p$ of the neighbors of $u$ to remain active. Therefore, the value of $\delta(n+1)$ (where $n+1$ represents a round number) is $\delta(n) \times q$. Therefore $\alpha$ decreases at a constant rate, which is what we wanted to show.

\end{proof}

\begin{cor}WHP, the number of communication rounds required to resolve Algorithm~\ref{alg:dgmm} for a random graph is $O(\log\Delta)$.\end{cor}

\end{proof} 

Therefore, because DGMM weights all edges and assigns nodes to the cover in a manner equivalent to GMM in $O(log \Delta)$ communication rounds for random graphs, Theorem~\ref{thm:dgmm-term} is proved.
\end{proof}


\subsection{Experimental Design and Results}

In this section we'll present the design and results of our experiments.

\section{Edge Coloring Problem}

In this section we'll present the approach for edge coloring.

\subsection{Distance-1 Algorithm}

\subsection{Modified Distance-1 Algorithm}


\section{Experimental Results}


\subsection{Distance 2 Algorithm}
\subsection{Experimental Results}

\section{Expansion to other problems and problems that cannot be solved in covering/covering}

\bibliographystyle{plain}
\bibliography{ijnc-apdcm-2012}


\end{document}

