\label{sec:dimaed-correct}
We here address the message complexity, termination, correctness, and solution quality of Algorithm~\ref{alg:edge-color}. We will first show that Algorithm~\ref{alg:edge-color} is likely to terminate in $O(\Delta)$ rounds, then that it will produce a correct coloring if it does terminate, and finally that the coloring produced will use no more than $2\Delta - 1$ colors.

\subsubsection{Message Complexity of Algorithm~\ref{alg:edge-color}}

Each node keeps, in some data structure, a list of it's neighbors and the colors that they are currently using. In our simulator, this is accomplished by using a list of hashes, where the id of each neighbor is the key to a hash, and the value of that hash is the list of colors that that particular neighbor is currently using. In the first round, these color lists are empty, as no edges have been colored yet.

As did DGMM, Algorithm~\ref{alg:edge-color} (DiMaEd), assumes a message passing model of computing, where each node may pass a message to each neighbor in each communication round, and communication rounds are assumed to be synchronized. Also, each computation round requires 3 communication rounds, first, invitations must be sent, second, responses must be sent, and finally, color updates must be sent.

An invitation from $u$ to $v$ to color edge $(u,v)$ contains three integers, the id of $u$, the id of $v$, and the intended color. This color is the lowest common indexed color between $u$ and $v$, which in the first round is 0, and in subsequent rounds can be found by comparing $u's$ current list of used colors and the current used colors of $v$, which is information that $u$ has in local storage.

A response from $v$ to $u$ contains three integers as well, the id of $v$, the id of $u$, and the index of the accepted color. 

The final message is the color update message, which contains the id of the node sending the update and the index of the color that it used in the current round. As each node can only color a single edge in a single round, this message contains two integers.

So the message complexity of DiMaEd is constant.

\subsubsection{Termination of Algorithm~\ref{alg:edge-color}}

We propose the following Lemma.
\begin{lemm}
\label{lem:edge-color-terminate}
Algorithm~\ref{alg:edge-color} is likely to terminate in $O(\Delta)$ rounds.
\end{lemm}

\input{prfs/prf-lemma-edge-color-terminate}

\subsubsection{Correctness of Algorithm~\ref{alg:edge-color}}

We propose the following Lemma.
\begin{lemm}
\label{lem:edge-color-correct}
Algorithm~\ref{alg:edge-color} produces a correct coloring. 
\end{lemm}

\begin{proof}[Proof of Lemma~\ref{lem:edge-color-correct}]

Assume that Algorithm~\ref{alg:edge-color} does not produce a correct coloring, There are two cases where this could occur: either there exists a node $v$ that uses some color twice, or there exist nodes $v,w$ that color the edge $(v,w)$ with different colors. 

A vertex colors an edge after negotiation with some neighbor. In order for an edge to be colored, a vertex $v$ must send an invitation to some neighbor $w$ to color $(v,w)$ with a specific color. Because we assume a message passing model, we assume that $w$ recieves this invitation. If $w$ responds to the invitation, $v$ assigns the color and $w$ assigns the color. In the message passing model, it is safe to assume that $v$ recieves the response from $w$. In order for $v$ to choose a different color than $w$ for $(v,w)$, $v$ would have to either color the edge without a response, which is contrary to the behavior of the vertex (line~\algref{alg:edge-color}{alglin:ec-receive-responses}), or $v$ must not receive the message, which is contrary to our model.

In the second case, a vertex could use the same color twice if it either issued or responded to an invitation to use a color twice. We know, however, that whenever an algorithm uses a color, that color is assigned to a list (lines~\algref{alg:edge-color}{alglin:ec-assign1},~\algref{alg:edge-color}{alglin:ec-assign2}). These colors are further removed from the list in each round (line~\algref{alg:edge-color}{alglin:ec-update-colors}).

If a vertex responded to or issued more than one invitation in a single round, it is possible that this conflict could occur, but this also contradicts the behavior of the algorithm of building a message containing a single id in either case.

Algorithm~\ref{alg:edge-color} therefore produces a correct coloring.
\end{proof}


\subsubsection{Solution Quality of Algorithm~\ref{alg:edge-color}}

We propose the following Lemma.
\begin{lemm}
\label{lem:edge-color-approximate}
Algorithm~\ref{alg:edge-color} will use $2\Delta - 1$ colors in the worst case. 
\end{lemm}

\begin{proof}[Discussion of Lemma~\ref{lem:edge-color-approximate}]

We begin by showing the worst case performance of Algorithm~\ref{alg:edge-color}. 

In each round that a node joins a pair, both nodes use the lowest common indexed color to color the edge between them. So in the first round, color 1 will be used for every edge in the matching, in the second round, color 1 or color 2, in the third 1,2, or 3, and so forth. 

So to model our worst possible case, we propose a node $w$ with the following characteristics. First, node $w$ has a degree of $\Delta$ and all of the neighbors of $w$ have a degree of $\Delta$. Second, $w$ does not participate in the matching in the first $\Delta-1$ rounds, but every neighbor of $w$ does. In this way we insure that $w$ cannot form an edge with any neighbor with an index of less than $\Delta-1$. 

In this case, $w$ will be forced to use an additional $\Delta$ colors to connect to each neighbor, and the total number of colors used will be $2\Delta-1$.
\end{proof}


Taken together, Lemmas~\ref{lem:edge-color-terminate}, \ref{lem:edge-color-correct}, and~\ref{lem:edge-color-approximate} give us the following theorem.

\begin{thm}
Algorithm~\ref{alg:edge-color} will produce a 2-approximate coloring in $O(\Delta)$ rounds in the typical case.
\label{thm:edge-color}
\end{thm}


This approximation bound is weak. We also made the following conjecture, which was tested experimentally and born out in some classes of graphs.

\begin{con}
\label{lem:edge-color-optimal}
Algorithm~\ref{alg:edge-color} uses $C \le \Delta + 1$ colors in the typical run. 
\end{con}

\begin{proof}[Discussion of Conjecture~\ref{lem:edge-color-optimal}]

A graph can certainly be colored with either $\Delta$ or $\Delta + 1$ colors. If a node $v$ were to be forced to use $\Delta + 2$ colors to color a graph, that would mean that there are two colors of index $\le \Delta$ which are being used by each neighbor of $v$ but not by $v$ itself.

In order for this to happen, there would need to be some round, or sequentially set of rounds, in which all of $v$'s neighbors formed a matching, and $v$ did not. We know from Proposition~\ref{lem:edge-color-terminate} that the odds of a node forming a match in a given round are greater than $\sfrac{1}{4}$, because the odds of a node being an invitor and recieving an invitation are approximately $\sfrac{1}{4}$. We can also easily calculate that the odds of a node being an invitor and sending a successful invitation are no greater than $\sfrac{1}{4}$, since there is a $\sfrac{1}{2}$ chance that a node $w$ will choose to send invitations and a $\sfrac{1}{2}$ chance that the neighbor $w$ sends an invitation to is an invitee.  

So the odds of a node forming a pair at all in a given round are $\sfrac{1}{x}$, $4 \ge x \ge 2$. 

For a given node to not form a pair while all of its neighbors do form pairs is therefore akin to the odds that in a fair coin toss, we first flip heads and then flip tails some arbitrary number of times in a row, or that in a simultaneous coin toss of some number of coins, one is heads while the rest are tails. 

We therefore expect our algorithm to behave well in the following sense: we should get conistent results with similar graphs, the algorithm should color with $\Delta$ or $\Delta + 1$ colors most of the time, and in no experimental case should we ever see the maximum $2\Delta - 1$ colors used.

\end{proof}


