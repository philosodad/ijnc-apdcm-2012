\label{ssb:algorithms-dgmm-performance}

Despite its overall simplicity, Algorithm~\ref{alg:dgmm} has a running time of $O(\log \Delta)$, which improves on the previous best result of $O(\log n)$ in \cite{1582746} without sacrificing the performance bound of two-optimality. 

The proof for this depends on two points, first, that Algorithm~\ref{alg:dgmm} is correct, and second, we need to prove that the algorithm progesses towards a solution at a logarithmic rate. 

To demonstrate the first point, we must show that our algorithm is equivalent in behavior to the sequential algorithm, which has already been shown to be correct\cite{Gonzalez1995129}. To do this we need to show that if each edge that is weighted in a given communication round were instead weighted sequentially in some arbitrary order, this would have no effect on the weightings assigned to each edge. If we show that, then because we use the same criteria for weighing edges as used by the sequential algorithm, and the sequential algorithm proceeds by weighting edges in an arbitrary order, our algorithm is equivalent.

More formally, we make the following propositions:

\begin{prp}
  \label{prop:dgmm-edge-order}
  Given a matching, a simultaneous weighting of the edges in that matching is equivalent to an arbitrary sequential weighting of the same edges.
\end{prp}


\begin{prp}
  \label{prop:dgmm-edge-match}
  DGMM produces a matching in each communication round.
\end{prp}

\begin{prp}
  \label{prop:dgmm-edge-once}
  DGMM weights every edge exactly once.
\end{prp}


A formal proof of these propositions follows.

  \begin{proof}[Proof of Proposition~\ref{prop:dgmm-edge-order}]
    By definition of a matching, no two edges in a matching share a vertex. Therefore, if an edge $e(u,v)$ is in the matching, no edge $e(u,i)$ is in the matching. 
    Take a matching \bMd\ in a graph $G(V,E)$. If we use the sequential algorithm to weight the edges in \bMd\ one after the other according to~\eqref{eqn:gmm}, it is obvious that no edge outside of \bMd\ will change. Since only edges outside of \bMd\ are used to assign weights to edges inside \bMd\, it does not matter what order the weights are assigned in, or whether the weight assignment occurs to all edges in \bMd\ simultaneously.
    
    Therefore Proposition~\ref{prop:dgmm-edge-order} is true.
  \end{proof}
  \begin{proof}[Proof of Proposition~\ref{prop:dgmm-edge-match}]
    Assume not, that is, assume that there are two edges $e(u,v) \text{ and } e(u,i)$ that are both updated during the same communication round. For this to happen, some compute node $u$ must form a partnership with two nodes $i$ and $v$. 
    
At the beginning of every communication round, each node makes an equally weighted random decision to either issue an invitation or wait for invitations. We consider these options by cases.

    Case One: Assume that $v$ issues invitations. If $v$ issues invitations, $v$ will choose a single unweighted edge $(v,u)$ and broadcast an invitation with the id of $u$ to all of its neighbors (Line~\algref{alg:dgmm}{alglin:dgmm-issue-invite}). $v$ then transitions to the \cWd\ state. In this state, the node gathers all responses issued by its neighbors, and updates an edge if a response is sent specifically to .

    So if two edges are weighted, $v$ must receive two responses.

    Responses are issued by nodes in the \cRd\ state. Each node in this state chooses a single invitation from its received invitations and responds to it. Since $v$ gets two responses, therefore, $v$ must have invited two separate nodes in this round. But $v$ only issues one invitation, so this is a contradiction.

    Case Two: Assume that $v$ receives invitations. A node which recieves invitations updates the edge corresponding to the invitation it recieved. Since $v$ is weighting two edges, $v$ must respond to multiple invitations in this round. However, $v$ only sends a single response message (Line~\algref{alg:dgmm}{alglin:dgmm-choose-invite}), which is a contradiction as well.
    Therefore, Proposition~\ref{prop:dgmm-edge-match} is true.
  \end{proof}
  \begin{proof}[Proof of Proposition~\ref{prop:dgmm-edge-once}]
    Because a node will only attempt to weight an unweighted edge, we know that no edge will be weighted more than once. If the proposition is false, it must be the case that some edge is not weighted.
    For an edge to be unweighted, both endpoints of the edge would have to halt (enter the \cDd\ state) before the edge is weighted. Nodes halt under two circumstances:
    \begin{enumerate}
    \item The node has joined the cover.
    \item A node's neighbors have all joined the cover.
    \end{enumerate}
    In the first case, the node will weight all of its unweighted edges to 0. In the second case, the node weights its own edges to 0 if the other endpoint is in the cover.
    Therefore, if the algorithm halts, all edges have been weighted once.
  \end{proof}
  
  Taken together, the three propositions give us the following Lemma,
  \begin{lem}
\label{lem:dgmm-edge}
  DGMM weights each edge once in a manner equivalent to GMM.
\end{lem}


  As we have indicated, prior work by Gonzalez has shown GMM to be 2-optimal, therefore we have the following corallary.
\begin{corr}\label{cor:dgmm-two}DGMM is 2-optimal.\end{corr}


Given that DGMM is correct and produces a 2-optimal cover, the second point that must be shown is that DGMM terminates in $O(\log{\Delta})$ communication rounds.

Intuitively, we look at this second point from the point of view of a single node. The algorithm itself terminates when every individual node reaches the ``Done'', or \cDd\ state. An individual node will reach this state under two circumstances, when either itself or all of it's neighbors join the cover.

It is the second condition that we are interested in, specifically, we want to know, for any random node, how many of it's neighbors will join the cover in any particular round. If some constant fraction of those neighbors will join the cover in a given round, than the progression of the algorithm will be logarithmic on $\Delta$. 

We use the following terminology in our proof of the time complexity of DGMM. We call a node {\em committed} if it has terminated execution, either by joining the cover or having all of its neighbors join the cover. A node which is not {\em committed} is {\em active}. The maximum degree of the graph is $\Delta$. A node's {\em active degree} is its number of unweighted edges, indicated by $\alpha(u)$ for some node $u$. The largest active degree in the graph is $\delta \le \Delta$. 

In order to show that a constant fraction of the neighbors of any given node will join the cover in any given round, we want to show that first, there is a constant probability of any node joining the cover in any given round, and that second, $\delta$ decreases by a constant fraction in each round. Formally, we have the following two propositions.

\begin{prp}
  \label{prop:dgmm-log-each}
  Each active node in the graph joins the cover with a constant probability in each round.
\end{prp}

\begin{prp}
  \label{prop:dgmm-log-alpha}
  With high probability, $\delta$ decreases by a constant fraction in each round.
\end{prp}


\begin{proof}[Proof of Proposition~\ref{prop:dgmm-log-each}]

We begin with a node, $w$, which is an active node in the graph. We want to show that the probability that $w$ will join the cover is constant. Let us assume that $w$ chooses to be a receiver, an event which happens with $P(0.5)$

We know that $w$ has some number of active neighbors, indicated by $\alpha(w)$. Each neighbor of $w$ also has some number of active neighbors. Each active degree is an integer between 1 and $\delta$. We can consider each node to have approximately the same active degree, which we indicate by $\alpha$. 

Each active neighbor of $w$ also chooses to be a sender or a receiver with equal probability. Following the assumption of equal distribution, there are $\sfrac{\alpha}{2}$ senders in the neighborhood of $w$, and each sender has $\alpha$ neighbors.

Each neighbor of $w$ that chooses to be a sender will, according to Line~\ref{alglin:dgmm-issue-invite} of Algorithm~\ref{alg:dgmm}, choose one neighbor which is active and send an invitation to that node. Because there are approximately $\sfrac{\alpha}{2}$ inviting neighbors of $w$, we can consider the probability that $w$ will recieve an invitation to be approximately equivalent to the probability that any event with a probability of $\sfrac{1}{n}$ will occur in $\sfrac{n}{2}$ independent trials for $n > 0$. 

Therefore, the probability that $w$ will recieve an invitation from an active neighbor is: \begin{equation}
1 - \left(\frac{n-1}{n}\right)^{\frac{n}{2}} > \lim_{0 \to \infty} 1 - \left(\frac{k-1}{k}\right)^{\frac{k}{2}} = 1 - \frac{1}{\sqrt{{\mathrm{e}}}} \approx 0.393
\end{equation}


If $w$ recieves at least one invitation, it will respond to exactly one invitation (\algref{alg:dgmm}{alglin:dgmm-choose-invite}), so the probability that $w$ will respond to an invitation from some active neighbor $v$ is exactly the same as the probability that $w$ will recieve an invitation from some active neighbor $v$. Therefore, if $w$ is a receiver, $w$ joins the cover with a probability of $1 - \sfrac{1}{\sqrt{{\mathrm{e}}}}$, which is constant. The probability of $w$ being a receiver at all is $\sfrac{1}{2}$, and the probability of $w$ joining the cover is also $\sfrac{1}{2}$, as we assume that weights are distributed arbitrarily through the graph. Therefore, the probability that $w$ will join the cover in any given round is constant.
\end{proof}

\begin{proof}[Proof of Proposition~\ref{prop:dgmm-log-alpha}]

We consider a node $u$, where $\alpha(u) = \delta$. $u$ may choose to be a sender or a receiver in this round, and $u$ may or may not join the matching. We do make the assumption that $u$ does not join the cover in this round.

$u$ has $\delta$ active neighbors. According to Proposition~\ref{prop:dgmm-log-each}, each of these neighbors joins the cover with constant probability. So in round one, we expect some constant percentage $p$ of the neighbors of $u$ to join the cover, and some constant percentage $q = 1-p$ of the neighbors of $u$ to remain active. Therefore, the value of $\delta(n+1)$ (where $n+1$ represents a round number) is $\delta(n) \times q$. Therefore $\alpha$ decreases at a constant rate, which is what we wanted to show.

\end{proof}


Taken together, Propositions~\ref{prop:dgmm-log-each} and~\ref{prop:dgmm-log-alpha} lead to the following Lemma.

\begin{lem}
\label{lem:dgmm-log}
Algorithm~\ref{alg:dgmm} (DMMW) terminates in $O(log \Delta)$ rounds with high probability for random connected graphs.
\end{lem}


\begin{proof}[Proof of Lemma~\ref{lem:dgmm-log}]
\begin{ldef}
A node is {\em committed} if it has joined the cover or if all of its neighbors have joined the cover.
\end{ldef}
\begin{ldef}
A node is {\em active} if it is not committed.
\end{ldef}
\begin{ldef}
$\Delta$ is the maximum degree of the Graph.
\end{ldef}

\begin{ldef}
The active degree of a node $u$ is the number of unweighted edges of $u$ indicated by $\alpha(u)$. $\alpha(u)$ is either 0 for committed nodes or equal to the number of active neighbors of $u$ for active nodes.
\end{ldef}
\begin{ldef}
$\delta$ is the largest active degree in the graph.
\end{ldef}

Lemma~\ref{lem:dgmm-log} can be restated in terms of the following propositions:
\begin{lprp}
\label{prop:dgmm-log-each}
Each active node in the graph joins the cover with a constant probability in each round.
\end{lprp}
\begin{lprp}
\label{prop:dgmm-log-alpha}
With high probability, $\delta$ decreases by a constant fraction in each round.
\end{lprp}
\begin{proof}[Proof of Proposition~\ref{prop:dgmm-log-each}]

We begin with a node, $w$, which is an active node in the graph. We want to show that the probability that $w$ will join the cover is constant. Let us assume that $w$ chooses to be a receiver, an event which happens with $P(0.5)$

We know that $w$ has some number of active neighbors, indicated by $\alpha(w)$. Each neighbor of $w$ also has some number of active neighbors. Each active degree is an integer between 1 and $\delta$. We can consider each node to have approximately the same active degree, which we indicate by $\alpha$. 

Each active neighbor of $w$ also chooses to be a sender or a receiver with equal probability. Following the assumption of equal distribution, there are $\sfrac{\alpha}{2}$ senders in the neighborhood of $w$, and each sender has $\alpha$ neighbors.

Each neighbor of $w$ that chooses to be a sender will, according to Line~\ref{alglin:dgmm-issue-invite} of Algorithm~\ref{alg:dgmm}, choose one neighbor which is active and send an invitation to that node. Because there are approximately $\sfrac{\alpha}{2}$ inviting neighbors of $w$, we can consider the probability that $w$ will recieve an invitation to be approximately equivalent to the probability that any event with a probability of $\sfrac{1}{n}$ will occur in $\sfrac{n}{2}$ independent trials for $n > 0$. 

Therefore, the probability that $w$ will recieve an invitation from an active neighbor is: \begin{equation}
1 - \left(\frac{n-1}{n}\right)^{\frac{n}{2}} > \lim_{0 \to \infty} 1 - \left(\frac{k-1}{k}\right)^{\frac{k}{2}} = 1 - \frac{1}{\sqrt{{\mathrm{e}}}} \approx 0.393
\end{equation}


If $w$ recieves at least one invitation, it will respond to exactly one invitation (\algref{alg:dgmm}{alglin:dgmm-choose-invite}), so the probability that $w$ will respond to an invitation from some active neighbor $v$ is exactly the same as the probability that $w$ will recieve an invitation from some active neighbor $v$. Therefore, if $w$ is a receiver, $w$ joins the cover with a probability of $1 - \sfrac{1}{\sqrt{{\mathrm{e}}}}$, which is constant. The probability of $w$ being a receiver at all is $\sfrac{1}{2}$, and the probability of $w$ joining the cover is also $\sfrac{1}{2}$, as we assume that weights are distributed arbitrarily through the graph. Therefore, the probability that $w$ will join the cover in any given round is constant.
\end{proof}

\begin{proof}[Proof of Proposition~\ref{prop:dgmm-log-alpha}]

We consider a node $u$, where $\alpha(u) = \delta$. $u$ may choose to be a sender or a receiver in this round, and $u$ may or may not join the matching. We do make the assumption that $u$ does not join the cover in this round.

$u$ has $\delta$ active neighbors. According to Proposition~\ref{prop:dgmm-log-each}, each of these neighbors joins the cover with constant probability. So in round one, we expect some constant percentage $p$ of the neighbors of $u$ to join the cover, and some constant percentage $q = 1-p$ of the neighbors of $u$ to remain active. Therefore, the value of $\delta(n+1)$ (where $n+1$ represents a round number) is $\delta(n) \times q$. Therefore $\alpha$ decreases at a constant rate, which is what we wanted to show.

\end{proof}

\begin{cor}WHP, the number of communication rounds required to resolve Algorithm~\ref{alg:dgmm} for a random graph is $O(\log\Delta)$.\end{cor}

\end{proof} 


Which, in conjunction with Lemma~\ref{lem:dgmm-edge} lead to the following theorem.

\begin{thm}
  Algorithm~\ref{alg:dgmm} (DGMM) will generate a 2-optimal cover in $O(log \Delta)$ communication rounds with high probability\footnote{The probability in this case is dependent on a fair coin toss. For the algorithm to solve in $O(\log{\Delta})$ rounds, the coins in a neighborhood should have a normal distribution, I.E., any given group of coins should have about the same number of heads as tails, and in any given neighborhood in any round, we should have about an equal number of inviters and receivers.} on random input.
\label{thm:dgmm-term}
\end{thm}


Which is what we wanted to derive.
