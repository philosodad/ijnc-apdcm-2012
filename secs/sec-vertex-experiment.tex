\label{sec:vertex-experiment}
In order to test our performance predictions, we designed a simulator to implement our algorithm (DGMM) and the algorithm of Koufoganiakkis and Young (K/Y). We generated random weighted graphs for $n=120, n=240, n=480, \text{and } n=960$, with average degrees of 3, 6, 12, 24, 48 and 96. 50 graphs were generated for each combination of sizes and average degree, for a total of 1200 graphs. DGMM and K/Y were used to produce a vertex cover for each graph, and the total weight and number of communication rounds were tallied and averaged for each of the 24 graph sizes. Our results showed that DGMM produced covers of approximately the same weight as K/Y in significantly fewer communication rounds in each case. Figures~\ref{plt:dgmm-comp} and \ref{plt:mwvc-rn} show our experimental results.\footnote{The source code for the simulator is free to use and available at this url http://bitbucket.org/philosodad/graph\_framework. The latest revision as of this writing is 06a7306cf497.}

\begin{figure}[htp]
\begin{center}
\begin{tikzpicture}
  \begin{axis}[xlabel=Average Degree, ylabel=Communication Rounds, legend style={at={(0.95,0.95)}, font=\footnotesize, label={[font=\footnotesize]left:K/Y}, anchor=north east}, legend columns=2, cycle list name={four-1-0}]
    \addplot+[grt] table [x=links, y=star-reg]{\runsone};
    \addplot+[grt] table [x=links, y=star-reg]{\runstwo};
    \addplot+[grt] table [x=links, y=star-reg]{\runsthree};
    \addplot+[grt] table [x=links, y=star-reg]{\runsfour};
    \addplot+[inv] table [x=links, y=mat-reg]{\runsone};
    \addplot+[inv] table [x=links, y=mat-reg]{\runstwo};
    \addplot+[inv] table [x=links, y=mat-reg]{\runsthree};
    \addplot+[inv] table [x=links, y=mat-reg]{\runsfour};
    \legend{(120),(240),(480),(960)}
  \end{axis}
  \begin{axis}[axis x line=none, axis y line=none, legend style={at={(0.05,0.425)}, font=\footnotesize, label={[font=\footnotesize]above left:DGMM}, anchor=north west}, legend columns=2, cycle list name={four-0-1}]
    \addplot+[inv] table [x=links, y=star-reg]{\runsone};
    \addplot+[inv] table [x=links, y=star-reg]{\runstwo};
    \addplot+[inv] table [x=links, y=star-reg]{\runsthree};
    \addplot+[inv] table [x=links, y=star-reg]{\runsfour};
    \addplot+[bls] table [x=links, y=mat-reg]{\runsone};
    \addplot+[bls] table [x=links, y=mat-reg]{\runstwo};
    \addplot+[bls] table [x=links, y=mat-reg]{\runsthree};
    \addplot+[bls] table [x=links, y=mat-reg]{\runsfour};
    \legend{,,,,(120),(240),(480),(960)}
  \end{axis}
\end{tikzpicture}

\caption{Rounds to resolve MWVC}
\label{plt:mwvc-rn}
\end{center}
\end{figure}

\begin{figure}[htp]
\begin{center}
\begin{tikzpicture}
  \begin{axis}[xlabel=Average Degree, ylabel=Total Weight, legend style={at={(.95,.69)}, label={[font=\footnotesize]left:K/Y}, font=\footnotesize, anchor=south east}, legend columns=2, cycle list name={four-1-0}]
    \addplot+[grt] table  [x=links, y=star-reg]{\averageone};
    \addplot+[grt] table  [x=links, y=star-reg]{\averagetwo};
    \addplot+[grt] table  [x=links, y=star-reg]{\averagethree};
    \addplot+[grt] table  [x=links, y=star-reg]{\averagefour};
    \addplot+[inv] table [x=links, y=mat-reg]{\averageone};
    \addplot+[inv] table [x=links, y=mat-reg]{\averagetwo};
    \addplot+[inv] table [x=links, y=mat-reg]{\averagethree};
    \addplot+[inv] table [x=links, y=mat-reg]{\averagefour};
    \legend{(120),(240),(480),(960)}
  \end{axis}
  \begin{axis}[axis x line=none,axis y line=none, legend style={at={(.95,.68)}, label={[font=\footnotesize]left:DGMM}, font=\footnotesize, anchor=north east}, legend columns=2, cycle list name={four-0-1}]
    \addplot+[inv] table  [x=links, y=star-reg]{\averageone};
    \addplot+[inv] table  [x=links, y=star-reg]{\averagetwo};
    \addplot+[inv] table  [x=links, y=star-reg]{\averagethree};
    \addplot+[inv] table  [x=links, y=star-reg]{\averagefour};
    \addplot+[bls] table [x=links, y=mat-reg]{\averageone};
    \addplot+[bls] table [x=links, y=mat-reg]{\averagetwo};
    \addplot+[bls] table [x=links, y=mat-reg]{\averagethree};
    \addplot+[bls] table [x=links, y=mat-reg]{\averagefour};
    \legend{,,,,(120),(240),(480),(960)}
  \end{axis}
\end{tikzpicture}


\caption{Average Weights For MWVC}
\label{plt:dgmm-comp}
\end{center}
\end{figure}


In Figure~\ref{plt:dgmm-comp}, the results for both algorithms are nearly identical, indicating that both algorithms found essentially the same covers for the input graphs. In Figure~\ref{plt:mwvc-rn}, the lower 4 curves indicate the number of rounds required to resolve DGMM, and the upper four the number of rounds required to resolve K/Y. The logarithmic growth confirms our experimental predictions. 

However, in order to account for differences between graphs of similar degree, we ran a second set of experiments on regular graphs. We constructed regular graphs with 256, 512, 1024, 2048, and 5096 nodes and degrees of 16, 32, 64 and 128. We ran the simulator on 20 graphs of each size. Figure~\ref{plt:dgmm-reg} shows our experimental results.

\begin{figure}[htp]
\begin{center}
\begin{tikzpicture}
  \begin{axis}[xlabel=Number of Nodes, ylabel=Communication Rounds, legend style={at={(0.95,0.20)}, font=\footnotesize, label={[font=\footnotesize]left:Degree}, anchor=north east}, legend columns=2, cycle list name={four-1-0}]
    \addplot+[grt] table [x=nodes, y=mat]{\regularone};
    \addplot+[grt] table [x=nodes, y=mat]{\regulartwo};
    \addplot+[grt] table [x=nodes, y=mat]{\regularthree};
    \addplot+[grt] table [x=nodes, y=mat]{\regularfour};
    \legend{(16),(32),(64),(128)}
  \end{axis}
\end{tikzpicture}

\caption{DGMM on Regular Graphs}
\label{plt:dgmm-reg}
\end{center}
\end{figure}


In this graph, each line represents the degree, with the number of nodes along the x-axis and the number of communication rounds on the y-axis. It is clear that the number of nodes in the graph does have an affect on the number of communication rounds for the algorithm. We cautiously postulate, based on our proof, that $O(\log{\Delta})$ remains the dominant term. 
