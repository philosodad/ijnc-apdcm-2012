\label{sec:dgmm-description}

Algorithm~\ref{alg:dgmm} is our distributed implementation of the 2-optimal minimum weighted vertex cover algorithm presented by Gonzalez \cite{Gonzalez1995129}. The Gonzalez algorithm, Generalized Maximal Matching for MWVC (or GMM), proceeds by selecting each edge in turn and choosing one of the endpoints of that edge to add to the cover. The sequential algorithm goes through each edge in turn and assigns the edge a weight according to~\eqref{eqn:gmm}.

\begin{align}
  \label{eqn:gmm}
  w(e(u,v)) = min 
  \begin{dcases} 
    w(u) - \sum_{i \ne v} w(e(u,i)) \\
    w(v) - \sum_{i \ne u} w(e(i,v)) 
  \end{dcases} \nonumber \\ 
\intertext{where } 
   w(e(u,i))\text{ or } w(e(i,v)) = 0 \nonumber
\intertext{for unweighted edges}
\end{align}


So if there are no previously weighted edges incident to either endpoint, the weight of the edge is $min(w(u),w(v))$. A vertex $u$ joins the cover when, for all it's weighted edges: \begin{equation}\sum_i w(e(u,i)) = w(u) \label{eqn:sat} \end{equation} When~\eqref{eqn:gmm} is applied to subsequent unweighted incident edges of $u$, the result will be 0. The algorithm terminates when each edge has been weighted. 

In GMM, every edge is examined exactly once. If no endpoints of the edge are in the cover, one endpoint will join the cover. Finally, all the edges are evaluated in arbitrary order. We explore this third point further in Section~\ref{ssb:algorithms-dgmm-performance}. 

Our distributed version of the algorithm chooses some disjoint set of edges (a matching) and assigns weights to those edges according to~\eqref{eqn:gmm}. We describe how each node transitions through the automata.\footnote{For the following description, we will substitute the symbol $\varpi(u)$ for the term $\sum_i w(e(u,i))$ in~\eqref{eqn:gmm}.\label{fn:varpi}}

\paragraph{\cCd\ (Choose) State} 
Each node that has not decided to join the cover or not begins the round in the \cCd\ state. In this step, each node chooses randomly whether transition to the \cId\ or \cLd\ state for the next step. 

\paragraph{\cId\ (Invite) State}
Each node $u$ in the \cId\ state chooses an unweighted edge $e(u,v)$ at random, and sends an invitation containing its own incident edge weight ($\varpi(u)$)\footnotemark[\value{footnote}], its own id, and the id of $v$, shown in line~\algref{alg:dgmm}{alglin:dgmm-issue-invite}. This broadcast is strictly local, that is, the neighbors of $u$ are the only possible choices for $v$, and the only nodes that receive the invitation. $u$ then transitions---either after communication or in synchrony with the transition to the next step for all nodes---to the \cWd\ state. 

\paragraph{\cLd\ (Listen) State} 
Each node $v$ in the \cLd\ state collects the invitations that contain an edge $e(i,v)$, where $i$ is a neighbor of $v$. An invitation contains precisely 3 values, all of which may be integers: the id of $i$, the id of $v$, and the current incident weight $varpi(i)$, represented by the notation $I_u^v\varpi(v)$ in Algorithm~\ref{alg:dgmm}. 

These nodes then transition to the \cRd\ state. Nodes that do not receive invitations also transition to the \cRd\ state, based on the universal synchrony assumed by the model. At this time each node in the graph is in either the \cRd\ or \cWd\ state. 

\paragraph{\cRd\ (Respond) State}
Each node $v$ in the \cRd\ state chooses at random from the invitations it collected in the \cLd\ state, determining which of the possible edges that it could weight will be weighted in this round. If $v$ did not receive any invitations it will not respond to any, but if it received at least one it will respond to precisely one. 

As mentioned previously, an invitation is a message $I_u^v\varpi(u)$, where $u$ is a neighbor of $v$ which sent the invitation and $\varpi(u)$ is the incident weight of $u$. $v$ constructs a response $R_v^u\varpi(v)$, containing it's own id ($v$), the id of $u$, and it's own incident weight ($\varpi(v)$). All nodes in the \cRd\ state, whether they issued a response or not, transition to the \cUd\ state. 

\paragraph{\cWd\ (Wait) State}
Each node $u$ in the \cWd\ state listens for responses to the invitations it sent (recall that the transition is $\cI \rightarrow \cW$). $u$ listens for a response $R_v^u\varpi(v)$. Because it sent one and only one invitation, $u$ can receive at most one response. Nodes in the \cWd\ state will transfer to the \cUd\ state synchronously according to the model. 

\paragraph{\cUd\ (Update) State}
To review, a node starts in the \cCd\ state and chooses randomly to transition to the \cId\ or \cLd\ state. \cId\ nodes transition $\cI \rightarrow \cW \rightarrow \cU$. A \cId\ node will issue a single invitation, and a \cWd\ node may receive a single response. Invitations and responses are strictly local, and contain exactly three integers of information, two node ids and one incident weight. 

\cLd\ nodes transition $\cL \rightarrow \cR \rightarrow \cU$. A \cLd\ node may receive a number of invitations, and if it receives any invitations it will respond to exactly one invitation.

So at this point, some of the \cId\ nodes have formed pairs with a neighboring \cLd\ node. This potential node pair can be denoted as $p(u,v)$. The collection of all $p$ for the graph is a matching on the graph. $u$ and $v$ have exchanged the following information: each node knows the id of the other and the incident weight of the other at the beginning of the round.

Recall that according to Equation~\eqref{eqn:gmm}, $u$ and $v$ will select the minimum incident weight between them and assign that weight to the edge $u,v$. No further communication is needed to weight the edge, both nodes have the same information and can independently select the minimum weight. 

Because the definition of incident weight for node $u$ is the weight of $u$ minus the sum of the edge weights of $u$, this makes the incident weight of at least one of $p(u,v)$ 0, and that node will join the cover. It should be mentioned that in the case of a tie, both nodes join the cover.

\paragraph{\cEd\ (Exchange) State}
Every node now transitions to the \cEd\ state, and broadcasts its current status to its neighbors. A node $u$ may have several unweighted edges at any time. For each of these unweighted edges $e(u,v)$, if $u$ receives a message that $v$ has joined the cover, $u$ will set the weight of $e(u,v)$ to 0 and remove $e(u,v)$ from the list of unweighted edges. Similarly, nodes that join the cover will assign a weight of zero to any unweighted edges. Again for brevity, the \cEd\ state is executed by implication in lines~\algref{alg:dgmm}{alglin:dgmm-state-E-W} and \algref{alg:dgmm}{alglin:dgmm-state-E}.

\paragraph{}Every node that is in the cover now transitions to the \cDd\ state, as does any node that has no unweighted edges remaining. Nodes that still have unweighted edges go back to the \cCd\ state and repeat the process. 
